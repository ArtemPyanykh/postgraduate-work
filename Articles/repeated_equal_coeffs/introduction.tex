\section{Введение}
В работе Б. Де Мейера и Х. Салей [\demeyer] рассматривается модель многошаговых торгов однотипными акциями. 
Торги ведут между собой два игрока.
Случайная цена акции может принимать два значения ($ 0 $ или $ m $).
Перед началом торгов случайный ход определяет цену акции на весь период торгов.
Выбранная цена сообщается первому игроку и не сообщается второму, при этом второй знает, что первый игрок --- инсайдер.
Оба игрока знают вероятность высокой цены акции.

На каждом шаге торогов оба игрока одновременно и независимо назначают некоторую цену за акцию.
Игрок предложивший большую цену покупает у второго акцию по названной цене. Предложенные цены объявляются игрокам в конце каждого хода. Игроки помнят предложенные цены на всех предыдущих этапах торгов.
Задачей игроков является максимизация стоимости своего портфеля.

Де Мейер и Салей сводят эту модель к повторяющейся игре с неполной информацией и, решая эту игру, описывают оптимальное поведение обоих игроков и ожидаемый выигрыш инсайдера.

В модели Мейера--Салей игроки могут делать произвольные вещественные ставки, но поскольку реальные торги проводятся в тех или иных денежных единицах, более реалистично предположить, что игроки могут делать только дискретные ставки, пропорциональные минимальной денежной единице.
 
В работе [\domansky] В. Доманским рассмотрен дискретный аналог модели Мейера--Салей, где игрокам разрешено делать только целочисленные ставки от $ 0 $ до $ m $.
Получено решение игры неограниченной продолжительности.
Нахождение явного решения для конечных игр остается открытой проблемой.
Для $ n $--шаговых игр в работе [\kreps] В. Крепс получено явное решение при $ m \leq 3 $. 
В работе [\sandomirskaya] М. Сандомирской и В. Доманским найдено явное решение одношаговой игры при произвольном натуральном значении $ m $.

В настоящей работе будет исследована модификация модели [\domansky], в которой наибольшая ставка определяет направление транзакции, но цена сделки определяется путем переговоров между игроками. 

В работе [\samuelson] Чаттерджи и Самуэльсон рассматривают модель двухстороннего аукциона со следующим правилом торгов: продавец и покупатель одновременно назначают цену на товар, $ s $ и $ b $ соответственно, при этом, в случае осуществления сделки, товар продается по цене $ p~=~kb + (1-k)s $, где $ 0 \leq k \leq 1  $. Параметр $ k $ при этом можно интерпретировать как переговорную силу покупателя. 

В данной работе мы полагаем $ k = \frac{1}{2} $ и исследуем оптимальные стратегии игроков и их выигрыши в бесконечной игре.