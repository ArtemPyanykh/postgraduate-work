\section{Введение}
Часто для описания эволюции цен на финансовых рынках используется вероятностная модель случайных блужданий. Происхождение этих случайных флюктуаций цен принято объяснять некоторыми экзогенными факторами. Однако гипотеза о полностью экзогенном происхождении этих колебаний не является удовлетворительной.

В работе [\demeyer] Б. де Мейер и Х. Салей с помощью упрощенной модели многошаговых торгов однотипными акциями продемонстрировали, что винеровская компонента в эволюции цен на акцию может быть следствием асимметричной информации у игроков. В модели Мейера--Салей торги ведут между собой два игрока.
Случайная цена акции может принимать два значения (низкое или высокое).
Перед началом торгов случайных ход определяет цену акции на весь период торгов, и выбранная цена сообщается первому игроку и не сообщается второму. 
Оба игрока знают вероятность высокой цены акции.

В модели Мейера--Салей игроки могут делать произвольные ставки, но поскольку реальные торги проводятся в тех или иных денежных единицах, более реалистично предположить, что игроки могут делать только дискретные ставки, пропорциональные минимальной денежной единице.
 
Дискретный аналог модели Мейера--Салей был рассмотрен В. Доманским в работе [\domansky]. В модели Доманского акция может иметь два возможных значения цены --- $ 0 $ и $ m $. Игроки могут делать целочисленные ставки от $ 0 $ до $ m $. Игрок, сделавший большую ставку, покупает у другого игрока одну акцию по заданной цене. В случае равных ставок транзакции не происходит. Установлено, что в соответствующей игре оптимальная стратегия инсайдера порождает случайное блуждание цен соверешнных сделок. Блуждание совершается по множеству допустимых ставок с поглошением в крайних точках. Поглощение происходит в тот момент, когда оценка вторым игроком цены акции совпадает с ее истинным значением.

В настоящей работе будет исследована модель, в которой наибольшая ставка определяет направление транзакции, но цена сделки определяется путем переговоров между игроками. 

В работе [\samuelson] Чаттерджи и Самуэльсон рассматривают модель двухстороннего аукциона со следующим правилом торгов: продавец и покупатель одновременно назначают цену на товар, $ s $ и $ b $ соответственно, при этом, в случае осуществления сделки, товар продается по цене $ p~=~kb + (1-k)s $, где $ 0 \leq k \leq 1  $. Параметр $ k $ при этом можно интерпретировать как переговорную силу покупателя. 

В данной работе мы полагаем $ k = \frac{1}{2} $ и исследуем оптимальные стратегии игроков и их выигрыши в бесконечной игре.