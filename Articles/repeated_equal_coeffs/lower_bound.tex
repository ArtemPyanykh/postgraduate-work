\section{Оценка снизу}
\label{sec:lower_bound}

Пусть на первом шаге первый игрок применяет 
$ \fpfs{k} = (\fpfss{H}, \fpfss{L}) $, где 
$ \fpfss{H} = (\fpfssa{H}{k}, \fpfssa{H}{k+1}) $ и 
$ \fpfss{L} = (\fpfssa{L}{k}, \fpfssa{L}{k+1}) $, где 
$ \fpfssa{S}{i} $ -- вероятность сделать ставку равную $ i $ в состоянии $ S \in \{ H, L \} $.
Т.е. при применении $ \fpfs{k} $ первый игрок делает ставки $ k $ и $ k + 1 $ с некоторыми заданными вероятностями. 

Также $ \fpfs{k} $ можно определить, если задать полные вероятности действий $ k $ и $ k + 1 $ равные $ q_k $ и $ q_{k+1} $, а также апостериорные вероятности $ p(S|i) $ состояния $ S $, если на первом шаге игрок сделал ставку $ i $. 
При этом вероятности $ \fpfssa{S}{i} $ легко рассчитываются по формуле Байеса
\[
    \fpfssa{S}{i} = \frac{p(S|i)q_i}{p(S)}.
\]
В дальнейшем мы будем пользоваться именно таким параметрическим заданием $ \fpfs{k} $.

\begin{utver}
\label{utver:lower_bound:general_step}
При использовании $ \fpfs{k} $ первый игрок гарантирует себе на первом шаге выигрыш:
\begin{equation}
\label{eq:lower_bound:general_step}
K_1(p, \fpfs{k}, j) = \begin{cases}
    2mp - j - k - q_{k+1}, &\, j < k,\\
    2mp(H|k+1)q_{k+1} - (2k+1)q_{k+1}, &\, j = k,\\
    (2k + 1)q_k - 2mp(H|k)q_k, &\, j = k+1,\\
    j + k - 2mp + q_{k+1}, &\, j > k + 1.
\end{cases}
\end{equation}
\end{utver}
\begin{proof}
По определению
\begin{equation}
\label{eq:payoff:stage}
    K_1(p, \fpfs{k}, j) = pA^H(\fpfss{H}, j) + (1-p)A^L(\fpfss{L}, j).
\end{equation}

Распишем выигрыш на первом шаге в зависимости от значения $ j $ действия второго игрока.
При $ j < k $ имеем:
\begin{align*}
  A^H(\fpfss{H}, j)
  &= \fpfssa{H}{k} (2m - j - k) + \fpfssa{H}{k+1} (2m - j - k - 1) = (2m - j - k) - \fpfssa{H}{k+1},\\
  A^L(\fpfss{L}, j)
  &= \fpfssa{L}{k} (-j - k) + \fpfssa{L}{k+1} (-j - k - 1) = - j - k - \fpfssa{L}{k+1}, \\
\end{align*}
Отсюда получаем, что
\begin{multline*}
  K_1(p, \fpfs{k}, j) 
  = p (2m - j - k) - p\fpfssa{H}{k+1} + 
      (1-p)(-j - k) - (1-p)\fpfssa{L}{k+1} = \\
  = 2mp - j - k - (p\fpfssa{H}{k+1} + (1-p)\fpfssa{L}{k+1}) = 2mp - j - k - q_{k+1}.
\end{multline*}
Аналогично при $ j = k $ имеем:
\begin{align*}
  &A^H(\fpfss{H}, k) = \fpfssa{H}{k+1} (2m - 2k - 1), \;
  A^L(\fpfss{L}, k) = \fpfssa{L}{k+1} (-2k - 1), \\
  &K_1(p, \fpfs{k}, k) = 2mp\fpfssa{H}{k+1} - (2k+1)q_{k+1} = 2mp(H|k+1)q_{k+1} - (2k+1)q_{k+1}
\end{align*}
Значения $ K_1(p, \fpfs{k}, j) $ при $ j \geq k + 1 $ получаются из соображений симметрии.
\end{proof}

Введем на $ [0, 1] $ разбиение $ P_k $ с точками 
\[
\begin{array}{lcr}
  \peven{k} = \frac{k}{m}, \, k = \overline{0,m} 
  & \text{и} & 
  \podd{k} = \frac{k}{m} + \frac{1}{2m}, \, k = \overline{0, m - 1}.
\end{array}
\]

Определим через $ \fpefs{k} $ такое действие первого игрока, что $ \fpefs{k}~=~\fpfs{k} $ с параметрами
\begin{equation*}
  p(H|k) = \frac{k}{m} - \frac{1}{2m}, \quad
  p(H|k+1) = \frac{k}{m} + \frac{1}{2m}, \quad
  q_k = q_{k+1} = \frac{1}{2}
\end{equation*}
Пусть также $ \fpefs{0} = 0,\, \fpefs{m} = m $ (т.е. если неопределенности нет, первый игрок использует минимаксную стратегию).
Аналогично, через $ \fpofs{k} $ определим такое действие первого игрока, что $ \fpofs{k}~=~\fpfs{k} $ с параметрами
\begin{equation*}
  p(H|k) = \frac{k}{m}, \quad
  p(H|k+1) = \frac{k + 1}{m}, \quad
  q_k = q_{k+1} = \frac{1}{2}
\end{equation*}

\begin{utver}
  При $ p = \peven{k} $ первый игрок может гарантировать себе выигрыш на первом шаге не менее $ 0 $, при $ p = \podd{k} $ первый игрок может гарантировать себе выигрыш не менее $ \frac{1}{2} $.
\end{utver}
\begin{proof}
  Действительно, пусть при $ p = \peven{k} $ первый игрок применяет $ \fpefs{k} $, а при $ p = \podd{k} $ применяет $ \fpofs{k} $. Тогда из утверждения \ref{utver:lower_bound:general_step} немедленно получаем справедливость данного утверждения.
\end{proof}

Заметим, что если $ p \in P_k $, то при применении $ \fpefs{k} $ и $ \fpofs{k} $ апостериорные вероятности также принадлежат $ P_k $.
Таким образом, определив действие первого игрока на первом шаге для произвольного значения вероятности $ p $, мы тем самым определим стратегию первого игрока в игре $ G_n^m(p) $ произвольной продолжительности.

В силу рекурсивной структуры игры $ G_n^m(p) $ можно выписать рекуррентную формулу для нижней оценки значения игры $ L_n^m(p) $ при $ p \in P_k $:

\begin{equation}
\label{eq:lower_bound:recurrence_finite}
\begin{aligned}[c]
&L_n^m\left(\frac{k+\frac{1}{2}}{m}\right) = 
    \frac{1}{2} + \frac{1}{2}\left(
        L_{n-1}^m\left(\frac{k}{m}\right) + 
        L_{n-1}^m\left(\frac{k + 1}{m}\right)
    \right) \\
&L_n^m\left(\frac{k+1}{m}\right) = 
    \frac{1}{2}\left(
        L_{n-1}^m\left(\frac{k+\frac{1}{2}}{m}\right) + 
        L_{n-1}^m\left(\frac{k+\frac{3}{2}}{m}\right)
    \right) \\
&L_n^m(0) = L_n^m(1) = 0.
\end{aligned}
\end{equation}

Устремив $ n $ в бесконечность получим рекуррентную формулу для нижней оценки значения игры $ G_\infty^m(p) $. Перепишем эту рекуррентную формулу в следующем виде:
\begin{equation}
\label{eq:lower_bound:recurrence_infinite}
\begin{aligned}
L_0 &= 0,\, L_{2m} = 0, \\
L_{2k + 1} &= \frac{1}{2} + \frac{1}{2} (L_{2k} + L_{2k + 2}), \, 
    k = \overline{0,m-1}\\
L_{2k + 2} &= \frac{1}{2} (L_{2k + 1} + L_{2k + 3}), \, 
    k = \overline{0,m-1}.
\end{aligned}
\end{equation}

Введем следующее обозначение:
\begin{equation}
A = \left.
\left(\begin{array}{ccccccc}
    1 & -\frac{1}{2} & 0 & \cdots & 0 & 0 & 0 \\
    -\frac{1}{2} & 1 & -\frac{1}{2} & \cdots & 0 & 0 & 0 \\
    \hdotsfor{7} \\
    0 & 0 & 0 & \cdots & -\frac{1}{2} & 1 & -\frac{1}{2} \\
    0 & 0 & 0 & \cdots & 0 & -\frac{1}{2} & 1 \\
\end{array}\right)
b = \left(\begin{array}{c}
\frac{1}{2} \\
0\\
\cdots\\
0 \\
\frac{1}{2}
\end{array}\right)
\right\} (2m - 1).
\end{equation}


Тогда \eqref{eq:lower_bound:recurrence_infinite} можно переписать следующим образом:
\[
\begin{aligned}
AL &= b, \\
L_0 &= L_{2m} = 0,
\end{aligned}
\]
где $ L = (L_1, L_2, \ldots, L_{2m-1})^T $.

% Refactored till this point

Известно, что системы $ Mx = F $ с трехдиагональной матрицей $ M $, имеющей следующую структуру:
\[
M = \left(\begin{array}{ccccccc}
c_1 & b_1 & 0 & \cdots & 0 & 0 & 0 \\
a_2 & b_2 & c_2 & \cdots & 0 & 0 & 0 \\
\hdotsfor{7} \\
0 & 0 & 0 & \cdots & a_{n-1} & c_{n-1} & b_{n-1} \\
0 & 0 & 0 & \cdots & 0 & a_n & c_n
\end{array}\right)
\]
можно решать методом прогонки, используя следующие формулы для прогоночных коэффициентов и значения переменного:
\begin{equation}
\label{eq:tridiagonal:formulas}
\begin{split}
    &x_i = \alpha_{i+1} x_{i+1} + \beta_{i+1}, \\
    &\alpha_{i+1} = -\frac{b_i}{c_i + a_i\alpha_i}, \\
    &\beta_{i+1} = \frac{F_i - a_i\beta_i}{c_i + a_i\alpha_i}, \\
\end{split}
\qquad
\begin{split}
    &x_n = \frac{F_n - a_n\beta_n}{c_n + a_n\alpha_n} \\
    &\alpha_2 = -\frac{b_1}{c_1} \\
    &\beta_2 = \frac{F_1}{c_1}
\end{split}
\end{equation}

Матрица $ A $ является трехдиагональной матрицей, в которой $ a_i = b_i = -\frac{1}{2}, \, c_i = 1 $. В этом случае мы можем получить явные выражения для прогоночных коэффициентов и значения $ L $.

\begin{utver}
\label{utver:tridiagonal:coefficients}
Прогоночные коэффициенты для матрицы $ A $ выражаются следующим образом:
\begin{align}
\label{eq:tridiagonal:alpha}
&\alpha_i = \frac{i-1}{i} \\
\label{eq:tridiagonal:beta}
&\beta_{2i} = \frac{2i}{4}, \,
\beta_{2i+1} = \frac{i^2}{2i+1}
\end{align}
\end{utver}
\begin{proof}
Проведем доказательство по индукции. 

Сначала для $ \alpha_i $. $ \alpha_2  = \frac{1}{2} / 1 = \frac{1}{2} $. База индукции проверена. Пусть теперь $ \forall i \leq n $ справедливо \eqref{eq:tridiagonal:alpha}. Докажем, что \eqref{eq:tridiagonal:alpha} справедливо при $ i = n + 1 $.
\[
    a_{n+1} = \frac{\frac{1}{2}}{1 - \frac{1}{2}\frac{n - 1}{n}}
    = \frac{1}{2} \frac{1}{\frac{n+1}{2n}} = \frac{n}{n+1}.
\]
Утверждение доказано для $ \alpha_i $. Аналогично для $ \beta_i $: 
$ \beta_2 = \frac{1}{2} $, 
$  \beta_3 = \frac{0 + 1/2 \cdot 1/2}{1 - 1/2 \cdot 1/2} = \frac{1}{3} $. База индукции проверена. Пусть $ \forall i \leq n $ справедливо \eqref{eq:tridiagonal:beta}. Докажем, что \eqref{eq:tridiagonal:beta} справедливо при $ i = n + 1 $.
\begin{align*}
    \beta_{2(n+1)+1} 
    &= \frac{
        \frac{1}{2} + \frac{1}{2} \frac{n^2}{2n+1}
    }{
        1 - \frac{1}{2} \frac{2n}{2n+1}
    }
    = \frac{1}{2} \frac{
        \frac{(n+1)^2}{2n+1}
    }{
        \frac{n+1}{2n+1}
    }
    = \frac{2(n+1)}{4} \\
    \beta_{2(n+1)+1}
    &= \frac{
        \frac{1}{2} \frac{2(n+1)}{4}
    }{
        1 - \frac{1}{2} \frac{2n+1}{2(n+1)}
    }
    = \frac{n+1}{4} \cdot 2 \cdot \frac{2(n+1)}{2n+3} = \frac{(n+1)^2}{2n+3}
\end{align*}

Утверждение полностью доказано.
\end{proof}

Из утверждения \ref{utver:tridiagonal:coefficients} и соображений симметрии, мы можем найти значение $ L_{2m - 1} = L_1 $. Действительно,
\begin{align}
\label{eq:lower_bound:recurrent_boundaries}
L_1 = L_{2m-1} &= 
\frac{
    \frac{1}{2} + \frac{1}{2}\frac{(m-1)^2}{2m-1}
}{
    1 - \frac{1}{2}\frac{2m-1-1}{2m-1}
} =
\frac{
    2m - 1 + (m-1)^2
}{
    4m - 2 - 2m + 2
} = \nonumber \\
& = \frac{2m - 1 + m^2  - 2m + 1}{2m} = \frac{m}{2}.
\end{align}

Так как 
\[
L_{2(m + 1)} = \frac{1}{2} (L_{2m+1} + L_{2(m+1) + 1}),
\]
то нам достаточно найти явный вид только для нечетных $ L_i $. Из \eqref{eq:tridiagonal:formulas} и утверждения \ref{utver:tridiagonal:coefficients} мы получаем, что
\begin{align}
\label{eq:lower_bound:recurrent_formula}
L_{2i+1} &= \alpha_{2(i+1)}(
    \alpha_{2(i+1)+1} L_{2(i+1)+1} + \beta_{2(i+1)+1}
    ) + \beta_{2(i+1)} = \nonumber \\
&= \alpha_{2(i+1)}\alpha_{2(i+1)+1} L_{2(i+1)+1} +
    \alpha_{2(i+1)}\beta_{2(i+1)+1} + \beta_{2(i+1)} = \nonumber \\
&= \frac{2i+1}{2i+3} L_{2i+3} + \frac{2(i+1)^2}{2i+3}.
\end{align}

\begin{utver}
\label{utver:lower_bound:solution}
Решение \eqref{eq:lower_bound:recurrent_boundaries}, \eqref{eq:lower_bound:recurrent_formula} дается следующей формулой:
\begin{equation}
\label{eq:lower_bound:recurrence_solution}
L_{2k+1} = \frac{m}{2} + mk - k^2 - k.
\end{equation}
\end{utver}
\begin{proof}
Для простоты перепишем \eqref{eq:lower_bound:recurrent_boundaries}, \eqref{eq:lower_bound:recurrent_formula} в следующем виде:
\begin{equation}
\label{eq:lower_bound:recurrent_formula_simplified}
f_{k+1} = \frac{2k+3}{2k+1} f_k - \frac{2(k+1)^2}{2k+1}, \quad f_0 = \frac{m}{2}.
\end{equation}

Решение этого конечно-разностного уравнения будем искать в следующем виде:
\[
f(k) = f^*(k) + Cf^O(k),
\]
где $ f^*(k) $ -- частное решение неоднородного уравнения, а $ f^O(k) $ -- решение однородного уравнения.

Сначала найдем $ f^O(k) $. Для $ f^O(k) $ выполнено
\begin{align*}
    f^O_1 &= \frac{3}{1} f_0 \\
    f^O_2 &= \frac{5}{3} f_1 \\
    &\cdots \\
    f^O_{k} &= \frac{2k+1}{2k-1} f^O_{k-1}
\end{align*}
Тогда $ f^O_k = (2k+1)f^O_0 $ и можно положить $ f^O(k) = 2k + 1 $. 

Перейдем к поиску частного решения. Будем искать его в виде $ f^*(k) = ak^2 + bk + c $. Тогда
\begin{align*}
    &2k\left[
        a(k+1)^2 + b(k+1) + c - ak^2 - bk - c)
    \right] + \\
    &\left[
        a(k+1)^2 + b(k+1) + c - 3 ak^2 - 3bk - 3c
    \right]
    = -2k^2 - 4k - 2 \\
    &2ak^2 + 4ak + a + b - 2c = -2k^2 - 4k - 2
\end{align*}
Положим $ a = -1, \, b = -1, \, c = 0 $. Тогда $ f^*(k) = -k^2 - k $. 
Таким образом $ f(k) = -k^2 - k + C(2k + 1) $. 

Из граничного условия найдем, что $ C = \frac{m}{2} $ и 
\[
f(k) = \frac{m}{2} + mk - k - k^2,
\]
а значит
\[
L_{2k+1} = \frac{m}{2} + mk - k - k^2.
\]

Утверждение доказано.
\end{proof}

Таким образом функция $ L^m(p) $ определена при
\[ p \in \left\{
    \frac{i}{m}, \, \frac{j+1/2}{m}, \,|\, i = \overline{0, m}, \, j = \overline{0, m-1}
\right\}.
\]
Для того, чтобы определить эту функцию при $ p $ лежащих внутри соответствующих отрезком нам нужная стратегия, гарантирующая первому игроку выигрыш, лежащий на прямой, соединяющей крайние точки этих отрезком. Данная стратегия дается следующим утверждением.

\begin{utver}
\label{utver:lower_bound:lottery}
Рассмотрим следующие случаи:
\begin{enumerate}
\item
Пусть 
$ p^a = \frac{k}{m} $,
$ p^b = \frac{k+1/2}{m} $,
$ p \in \left(p^a, p^b\right) $, 
$ k = \overline{0, m-1} $. 
Тогда определим первый шаг $ \sigma^{k}_{lot} $, как $ \sigma^k_1 $ с 
$ q_k = 2k + 1 - 2mp $, 
$ q_{k+1} = 2(pm - k) $, 
$ p(H|k) = \frac{k}{m} $,
$ p(H|k+1) = \frac{k+1/2}{m} $.

\item
Пусть 
$ p^a = \frac{k+1/2}{m} $,
$ p^b = \frac{k+1}{m} $,
$ p \in \left(p^a, p^b\right) $, 
$ k = \overline{0, m-1} $. 
Тогда определим первый шаг $ \sigma^{k}_{lot} $, как $ \sigma^k_1 $ с 
$ q_k = 2(k + 1 - mp) $, 
$ q_{k+1} = 2mp - 2k - 1 $, 
$ p(H|k) = \frac{k+1/2}{m} $,
$ p(H|k+1) = \frac{k+1}{m} $.
\end{enumerate}

В обоих случаях шаг $ \sigma^k_{lot} $ гарантирует первому игроку неотрицательный одношаговый выигрыш. Если на последующих шагах игрок будет применять $ \sigma^k_{eq} $, то он гарантирует себе выигрыш $ L^m(p) $, лежащий на прямой, соединяющей $ L^m(p^a) $ и $ L^m(p^b) $.
\end{utver}
\begin{proof}
Проведем доказательство для случая 1. Доказательство для случая 2 проводится аналогично.

Из утверждения \ref{utver:lower_bound:general_step} подстановкой соответствующих переменных получаем:
\begin{itemize}
\item
    $ j < k $: 
        $ K_1(p, \sigma^k_{lot}, j) = 2mp - j - k - 2mp + 2k = k - j \geq 1 $.
\item
    $ j = k $:
        $ K_1(p, \sigma^k_{lot}, j) = 2m\frac{k+1/2}{m}q_{k+1} - (2k+1)q_{k+1} = 0 $.
\item
    $ j = k + 1 $:
        $ K_1(p, \sigma^k_{lot}, j) = (2k + 1)q_k - 2m\frac{k}{m}q_k = q_k > 0 $.
\item
    $ j > k + 1 $:
        $ K_1(p, \sigma^k_{lot}, j) = j + k - 2mp + 2mp - 2k = j - k \geq 2 $.
\end{itemize}

Мы получили, что $ K_1(p, \sigma^k_{lot}, j) \geq 0 $. 

Далее заметим, что после применения $ \sigma^k_{lot} $ апостериорные вероятности принадлежат множеству
\[
    \left\{
        \frac{i}{m}, \, \frac{j+1/2}{m}, \,|\, i = \overline{0, m}, \, j = \overline{0, m-1}
    \right\}
\].
Это означает, что при применении на последующих шагах $ \sigma^i_{eq} $ игрок гарантирует себе 
$ L^m(p) = q_k L^m(p^a) + q_{k+1} L^m(p^b) $. Из того, что 
\[
    q_k = \frac{p^b - p}{p^b - p^a}, \quad q_{k+1} = \frac{p - p^a}{p^b - p^a} 
\]
следует, что $ L^m(p) $ лежит на прямой, соединяющей $ L^m(p^a) $ и $ L^m(p^b) $. Доказательство закончено.
\end{proof}



\begin{lemma}
\label{lemma:lower_bound:function}
Функция $ L^m(p) $ является кусочно-линейной функцией, состоящей из $ m $ линейный сегментов, и полностью определяется своими значениями в следующих точках:
\begin{eqnarray*}
& L^m(0) = L^m(1) = 0, \\
& L^m\left(\frac{k+1/2}{m}\right) 
    = \frac{m}{2} + mk - k^2 - k, 
        \, k = \overline{0, m - 1}
\end{eqnarray*}
\end{lemma}
\begin{proof}
Справедливость данной леммы непосредственно следует из утверждений \ref{utver:lower_bound:solution} и \ref{utver:lower_bound:lottery}.
\end{proof}