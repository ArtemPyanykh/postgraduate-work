\section{Значение игры $ G_\infty^m(p) $}
\begin{theorem}
\label{theorem:infinite_game:value}
Игра $ G_\infty^m(p) $ имеет значение 
\[
V_\infty^m(p) = H^m(p) = L^m(p).
\]

Оптимальная стратегия первого игрока $ \sigma^* $ дается утверждениями \ref{utver:first_player:strategy:extreme_points} и \ref{utver:lower_bound:lottery}.

Оптимальная стратегию второго игрока $ \tau^* $ описывается следующим образом: на каждом шаге игры при $ p \in \left( \podd{k-1}, \podd{k+1} \right] $ второй игрок применяет $ \tau^k $.
\end{theorem}
\begin{proof}
Непосредственно из лемм \ref{lemma:upper_bound:function} и \ref{lemma:lower_bound:function} следует, что
$ \forall \sigma \in \Sigma, \, \tau \in \Tau $, где $ \Sigma $ и $ \Tau $ -- множество стратегий первого и второго игроков соответственно, справедливо
\[
  K_\infty^m(p, \sigma, \tau^*)
\leq K_\infty^m(p, \sigma^*, \tau^*) = H^m(p) = L^m(p) \leq 
  K_\infty^m(p, \sigma^*, \tau).
\]
А это значит, что $ \sigma^* $ и $ \tau^* $ -- оптимальные стратегии, и значение игры $ V_\infty^m(p) = H^m(p) = L^m(p) $.
\end{proof}