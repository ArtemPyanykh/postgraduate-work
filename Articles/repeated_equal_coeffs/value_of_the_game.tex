\section{Значение игры $ G_\infty^m(p) $}
\begin{theorem}
\label{theorem:infinite_game:value}
Игра $ G_\infty^m(p) $ имеет значение 
\[
V_\infty^m(p) = L^m(p) = H^m(p).
\]

Оптимальная стратегия первого игрока $ \sigma^* $ описывается следующим образом: на каждом шаге игры при 
\[
  p \in \left\{
    \frac{k+1/2}{m}, \, \frac{k+1}{m} \,|\, k = \overline{0, m-1}
  \right\}
\]
первый игрок применяет соответствующую $ \sigma^k_{eq} $; при
\[
  p \in \left(
    \frac{k}{m}, \frac{k+1/2}{m}
  \right),\,
  k = \overline{0, m-1},
\]
или
\[
  p \in \left(
    \frac{k+1/2}{m}, \frac{k+1}{m}
  \right),\,
  k = \overline{0, m-1},
\]
игрок применяет соответствующую $ \sigma^k_{lot} $.

Оптимальная стратегию второго игрока $ \tau^* $ описывается следующим образом: при
\[
  p \in \left(
    \frac{k-1/2}{m}, \frac{k+1/2}{m}
  \right]
\]
второй игрок применяет $ \tau^k $.
\end{theorem}
\begin{proof}
Непосредственно из лемм \ref{lemma:upper_bound:function} и \ref{lemma:lower_bound:function} следует, что
$ \forall \sigma \in \Sigma, \, \tau \in T $, где $ \Sigma $ и $ T $ -- множество стратегий первого и второго игроков соответственно, справедливо
\[
  K_\infty^m(p, \sigma, \tau^*)
\leq K_\infty^m(p, \sigma^*, \tau^*) = H^m(p) \leq 
  K_\infty^m(p, \sigma^*, \tau).
\]
А это значит, что $ \sigma^* $ и $ \tau^* $ -- оптимальные стратегии, и значение игры $ V_\infty^m(p) = H^m(p) $.
\end{proof}
