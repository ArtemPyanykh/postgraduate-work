\section{Оценка сверху}
\label{sec:upper_bound}
Рассмотрим следующую чистую стратегию второго игрока\\
$ \tau^k $:
\begin{equation}
\label{eq:strategy:sp}
\begin{aligned}
\tau_1^k &= k \\
\tau_t^k(i_{t-1}, j_{t-1}) &= \begin{cases}
    j_{t-1} - 1, &\, i_{t-1} < j_{t-1} \\
    j_{t-1}, &\, i_{t-1} = j_{t-1} \\
    j_{t-1} + 1, &\, i_{t-1} > j_{t-1}
\end{cases}
\end{aligned}
\end{equation}

\begin{prop}
При применении стратегии \eqref{eq:strategy:sp} в игре $ G_n^m(p) $ второй игрок может гарантировать себе выигрыш не более:
\begin{eqnarray}
\label{eq:upper_bound:l}
h_n^L(\tau^k) &=& \sum_{t=0}^{n-1}
    (2(k - t) - 1)^+,\\
\label{eq:upper_bound:h}
h_n^H(\tau^k) &=& \sum_{t=0}^{n-1}
    (2(m - k - t) - 1)^+,
\end{eqnarray}
в состояниях $ L $ и $ H $ соответственно.
\end{prop}
\begin{proof}
Проведем доказательство по индукции.
Для $ h_n^L(\tau^k) $:

\[ 
h_1^L(\tau^k) = \max_{i \in I} a_{ik}^L = \max(0, 2k - 1).
\]
Таким образом база индукции для $ h_n^L(\tau^k) $ проверена. 

Пусть $ \forall t \leq n $ выполнено \eqref{eq:upper_bound:l}. Докажем, что \eqref{eq:upper_bound:l} выполняется при $ t=n+1 $.
\[
h_{n+1}^L(\tau^k) = \max_{i \in I} (a_{ik}^L + h_n^L(\tau^{c(i)})),
\]
где
\[ 
c(i) = \begin{cases}
    k - 1, &\, i < k \\
    k, &\, i = k \\
    k + 1, &\, i > k.
\end{cases}
\]

Рассмотрим значение выигрыша в зависимости от действия $ i $ первого игрока:
\begin{itemize}
\item
    $ i < k $:
    \begin{align*}
    h_{n+1}^L(\tau^k) &= 2k - 1 + h_n^L(\tau^{k-1}) = \\
    &= 2k - 1 + \sum_{t = 0}^{n-1}(2(k-1-t)-1)^+ = \\
    &= \sum_{t = 0}^n (2(k-t) - 1)^+
    \end{align*}
\item
    $ i = k $:
    \[
    h_{n+1}^L(\tau^k) = h_n^L(\tau^k) \leq \sum_{t=0}^n(2(k-t)-1)^+.
    \]
\item
    $ i > k $:
    \begin{align*}
    h_{n+1}^L(\tau^k) &= -2k - 1 + h_n^L(\tau^{k+1}) = \\
    &= -2k - 1 + \sum_{t = -1}^{n-2} (2(k-t)-1)^+ \\
    &\leq \sum_{t=0}^n(2(k-t)-1)^+.
    \end{align*}
    Индуктивный переход обоснован, и \eqref{eq:upper_bound:l} доказано.
    Аналогично для \eqref{eq:upper_bound:h}.
    \[
    h_1^H(\tau^k) = \max_{i \in I} a_{ik}^H = \max(0, 2(m - k) - 1).
    \]
    База индукции для $ h_n^H(\tau^k) $ проверена.
    Пусть $ \forall t \leq n $ выполнено \eqref{eq:upper_bound:h}. Докажем, что \eqref{eq:upper_bound:h} выполняется при $ t = n + 1 $.
    \[
    h_{n+1}^H(\tau^k) = \max_{i \in I}(a_{ik}^H + h_n^H(\tau^{c(i)}).
    \]
\item
    $ i > k $:
    \begin{align*}
    h_{n+1}^H(\tau^k) &= 2m - 2k - 1 + h_n^H(\tau^{k-1}) = \\
    &= 2(m-k)- 1 + \sum_{t = 0}^{n-1}(2(m-k-1-t)-1)^+ = \\
    &= \sum_{t = 0}^n (2(m-k-t) - 1)^+
    \end{align*}
\item
    $ i = k $:
    \[
    h_{n+1}^H(\tau^k) = h_n^H(\tau^k) \leq \sum_{t=0}^n(2(m-k-t)-1)^+.
    \]
\item
    $ i < k $:
    \begin{align*}
    h_{n+1}^H(\tau^k) &= 2k - 1 - 2m + h_n^H(\tau^{k+1}) = \\
    &= 2k - 1 - 2m + \sum_{t = -1}^{n-2} (2(m-k-t)-1)^+ \\
    &\leq \sum_{t=0}^n(2(m-k-t)-1)^+.
    \end{align*}
\end{itemize}

Таким образом индуктивный переход обоснован и утверждение полностью доказано.
\end{proof}

Можно заметить, что при $ t > m $ значения \eqref{eq:upper_bound:l} и \eqref{eq:upper_bound:h} стабилизируются. Таким образом, справедливо следующее

\begin{prop}
Для значения бесконечной игры справедливо следующее неравенство:
\begin{equation}
\label{eq:upper_bound:value}
V_\infty^m(p) 
<= 
H^m(p) = \min_{j \in J}
    (p(m-j)^2 + (1-p)j^2).
\end{equation}
\end{prop}
\begin{proof}

\begin{align*}
h_{m+1}^H(\tau^j) 
&= \sum_{t=0}^m (2(m-j-t)-1)^+ = \\
&= (2 \cdot 0 - 1)^+ + (2 \cdot 1 - 1)^+ + \ldots + (2(m-j) - 1)^+ = \\
&= (m-j)(m-j+1) - (m-j) = (m-j)^2
\\
h_{m+1}^L(\tau^j)
&= \sum_{t=0}^m (2(j-t)-1)^+ = \\
&= (2 \cdot 0 - 1)^+ + (2 \cdot 1 - 1)^+ + \ldots + (2 \cdot j - 1)^+ = \\
&= j(j+1) - j = j^2
\\
H^m(p) 
&= \min_{j \in J} \left(
    p h_{m+1}^H(\tau^j) + (1-p) h_{m+1}^L(\tau^j)
\right) \\
&= \min_{j \in J} \left(
p (m - j)^2 + (1-p)j^2
\right).
\end{align*}

\end{proof}

\begin{lemma}
\label{lemma:upper_bound:function}
Функция $ H^m(p) $ является кусочно-линейной функцией, состоящей из $ m $ линейных сегментов, и полностью определяется своими значениями в следующих точках:
\begin{eqnarray*}
& H^m(0) = H^m(1) = 0, \\
& H^m\left(\frac{k+1/2}{m}\right) 
    = \frac{m}{2} + mk - k^2 - k, 
        \, k = \overline{0, m - 1}
\end{eqnarray*}
\end{lemma}
\begin{proof}
Пусть $ \omega(j) = p(m-j)^2 + (1-p)j^2 $. 
\[
\omega'(j) = 2(j-pm), \quad
\omega''(j) = 2.
\]
Т.е. минимум $ \omega(j) $ достигается при $ j = pm $. 
Тогда при $ p \in \left( \frac{k - 1/2}{m}, \frac{k+1/2}{m} \right] $ минимум $ p(m-j)^2 + (1-p)j^2 $ достигается при $ j = k $. 
Таким образом, показано, что $ H^m(p) $ является кусочно-линейной функцией, которая полностью определяется своими значениями в точках 
$ p = \frac{k+1/2}{m}, \, k = \overline{0, m-1} $. Найдем значение $ H^m(p) $ при $ p = \frac{k+1/2}{m} $:
\begin{align*}
H^m \left(\frac{k+1/2}{m}\right)
&= \frac{1}{m} \left(
    (k+1/2)(m - k)^2 + (m-k-1/2)k^2
\right) = \\
&= \frac{1}{m} \left(
    m^2/2 + m^2 k - mk^2 - mk
\right) = \\
&= \frac{m}{2} + mk - k^2 - k.
\end{align*}
Лемма полностью доказана.
\end{proof}