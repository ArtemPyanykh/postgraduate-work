\section{Описание}
Рассмотрим следующую игру.
Есть две платежные матрицы:
\begin{eqnarray*}
A^L(i, j) &=& \frac{1}{m}\begin{cases}
\frac{i+j}{2}, &\, i < j\\
0, &\, i = j\\
-\frac{i+j}{2}, &\, i > j
\end{cases}
\\
A^H(i, j) &=& \frac{1}{m}\begin{cases}
\frac{i+j}{2} - m, &\, i < j\\
0, &\, i = j\\
m - \frac{i+j}{2}, &\, i > j,
\end{cases}
\end{eqnarray*}

На первом шаге случай выбирает $ S \in \{H, L\} $, причем вероятности выбора $ H $ и $ L $ равны $ p(H) = p $ и $ p(L) = (1 - p) $ соответственно. Первый игрок осведомлен о выборе случая, второй игрок знает только вероятностное распределение на $ \{ H, L\} $.
После этого на протяжении $ n \leq \infty $ шагов игроки играют в игру $ A^S $.

Множество чистых стратегий первого и второго игроков 
$ i \in I = \{0, 1, \ldots, m\}$,
$ j \in J = \{0, 1, \ldots, m\} $.
Выигрыш каждого из игроков равен суммарному выигрышу за $ n $ шагов.

Для упрощения вычислений умножим платежные матрицы на $ 2m $:
\begin{eqnarray}
\label{eq:payoff_matrix_h}
A^L(i, j) &=& \begin{cases}
  i + j, &\, i < j\\
  0, &\, i = j\\
  -i - j, &\, i > j
\end{cases}
\\
\label{eq:payoff_matrix_l}
A^H(i, j) &=& \begin{cases}
  i + j - 2m, &\, i < j\\
  0, &\, i = j\\
  2m - i - j, &\, i > j
\end{cases}
\end{eqnarray}

В дальнейшем будем рассматривать повторяющуюся игру именно с матрицами \eqref{eq:payoff_matrix_h} и \eqref{eq:payoff_matrix_l}, которую назовем $ G_n^m(p) $. При применении первым игроков смешанной стратегий 
$ \sigma = (\sigma_1, \sigma_2, \ldots, \sigma_n) $, где 
$ \sigma_t = (\sigma_t^L, \sigma_t^H)$, 
а вторым игроком смешанной стратегии 
$ \tau = (\tau_1, \tau_2, \ldots, \tau_n) $, выигрыш равен
\begin{equation}
\label{eq:value_of_game_n}
K_n^m(p, \sigma, \tau) = \sum_{t=1}^n
    \left(
        pA^H(\sigma_t^H, \tau_t) + (1 - p)A^L(\sigma_t^L, \tau_t)
    \right).
\end{equation}